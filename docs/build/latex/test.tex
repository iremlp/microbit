%% Generated by Sphinx.
\def\sphinxdocclass{report}
\documentclass[letterpaper,10pt,french]{sphinxmanual}
\ifdefined\pdfpxdimen
   \let\sphinxpxdimen\pdfpxdimen\else\newdimen\sphinxpxdimen
\fi \sphinxpxdimen=.75bp\relax

\PassOptionsToPackage{warn}{textcomp}
\usepackage[utf8]{inputenc}
\ifdefined\DeclareUnicodeCharacter
 \ifdefined\DeclareUnicodeCharacterAsOptional
  \DeclareUnicodeCharacter{"00A0}{\nobreakspace}
  \DeclareUnicodeCharacter{"2500}{\sphinxunichar{2500}}
  \DeclareUnicodeCharacter{"2502}{\sphinxunichar{2502}}
  \DeclareUnicodeCharacter{"2514}{\sphinxunichar{2514}}
  \DeclareUnicodeCharacter{"251C}{\sphinxunichar{251C}}
  \DeclareUnicodeCharacter{"2572}{\textbackslash}
 \else
  \DeclareUnicodeCharacter{00A0}{\nobreakspace}
  \DeclareUnicodeCharacter{2500}{\sphinxunichar{2500}}
  \DeclareUnicodeCharacter{2502}{\sphinxunichar{2502}}
  \DeclareUnicodeCharacter{2514}{\sphinxunichar{2514}}
  \DeclareUnicodeCharacter{251C}{\sphinxunichar{251C}}
  \DeclareUnicodeCharacter{2572}{\textbackslash}
 \fi
\fi
\usepackage{cmap}
\usepackage[T1]{fontenc}
\usepackage{amsmath,amssymb,amstext}
\usepackage[french]{babel}
\usepackage{times}
\usepackage[Sonny]{fncychap}
\ChNameVar{\Large\normalfont\sffamily}
\ChTitleVar{\Large\normalfont\sffamily}
\usepackage{sphinx}

\usepackage{geometry}

% Include hyperref last.
\usepackage{hyperref}
% Fix anchor placement for figures with captions.
\usepackage{hypcap}% it must be loaded after hyperref.
% Set up styles of URL: it should be placed after hyperref.
\urlstyle{same}
\addto\captionsfrench{\renewcommand{\contentsname}{Projets}}

\addto\captionsfrench{\renewcommand{\figurename}{Fig.}}
\addto\captionsfrench{\renewcommand{\tablename}{Tableau}}
\addto\captionsfrench{\renewcommand{\literalblockname}{Code source}}

\addto\captionsfrench{\renewcommand{\literalblockcontinuedname}{suite de la page précédente}}
\addto\captionsfrench{\renewcommand{\literalblockcontinuesname}{suite sur la page suivante}}

\addto\extrasfrench{\def\pageautorefname{page}}

\setcounter{tocdepth}{1}



\title{Ressources pour la carte Micro:bit}
\date{mai 09, 2018}
\release{1}
\author{IREM Marseille}
\newcommand{\sphinxlogo}{\vbox{}}
\renewcommand{\releasename}{Version}
\makeindex

\begin{document}

\maketitle
\sphinxtableofcontents
\phantomsection\label{\detokenize{index::doc}}


par le groupe InEFLP de l’IREM de Marseille


\chapter{Projets à réaliser}
\label{\detokenize{index:documentation-micro-bit}}\label{\detokenize{index:projets-a-realiser}}

\section{Températures}
\label{\detokenize{projets/temperature:projettemp}}\label{\detokenize{projets/temperature::doc}}\label{\detokenize{projets/temperature:temperatures}}

\subsection{Description}
\label{\detokenize{projets/temperature:description}}
\begin{sphinxadmonition}{note}{\label{projets/temperature:index-0}À faire:}
capture d’écran / gif animée.
\end{sphinxadmonition}


\subsubsection{Exemple(s) d’utilisation}
\label{\detokenize{projets/temperature:exemple-s-d-utilisation}}

\paragraph{Projet Températures - Exemple d’utilisation : un Escape game}
\label{\detokenize{projets/temperature-exemple-escape::doc}}\label{\detokenize{projets/temperature-exemple-escape:projet-projettemp-exemple-d-utilisation-un-escape-game}}
Nous avons utilisé le projet {\hyperref[\detokenize{projets/temperature:projettemp}]{\sphinxcrossref{\DUrole{std,std-ref}{Températures}}}} pour un escape game proposé en stage.
\begin{itemize}
\item {} 
diaporama d’acceuil : \sphinxurl{http://url.univ-irem.fr/temp}

\item {} 
page de formation : \sphinxurl{http://url.univ-irem.fr/algo1718-temp}

\end{itemize}


\subsection{Réalisation}
\label{\detokenize{projets/temperature:realisation}}

\subsubsection{Projet Températures - Fabriquer}
\label{\detokenize{projets/temperature-fabriquer:projet-projettemp-fabriquer}}\label{\detokenize{projets/temperature-fabriquer::doc}}
Nous détaillons ici comment fabriquer et assembler
le matériel nécessaire à la réalisation du projet
{\hyperref[\detokenize{projets/temperature:projettemp}]{\sphinxcrossref{\DUrole{std,std-ref}{Températures}}}}.

\begin{sphinxadmonition}{note}{\label{projets/temperature-fabriquer:index-0}À faire:}
tout faire.
\end{sphinxadmonition}


\subsubsection{Projet Températures - Coder}
\label{\detokenize{projets/temperature-coder:projet-projettemp-coder}}\label{\detokenize{projets/temperature-coder::doc}}
Nous détaillons ici le code nécessaire à la réalisation
du projet {\hyperref[\detokenize{projets/temperature:projettemp}]{\sphinxcrossref{\DUrole{std,std-ref}{Températures}}}}.

\begin{sphinxadmonition}{note}{\label{projets/temperature-coder:index-0}À faire:}
tout faire.
\end{sphinxadmonition}


\section{Boîte fermée}
\label{\detokenize{projets/boite-fermee:boite-fermee}}\label{\detokenize{projets/boite-fermee::doc}}\label{\detokenize{projets/boite-fermee:projetboite}}

\subsection{Description}
\label{\detokenize{projets/boite-fermee:description}}
\begin{sphinxadmonition}{note}{\label{projets/boite-fermee:index-0}À faire:}
capture d’écran / gif animée
\end{sphinxadmonition}


\subsubsection{Exemple(s) d’utilisation}
\label{\detokenize{projets/boite-fermee:exemple-s-d-utilisation}}

\paragraph{Projet Boîte fermée - Exemple d’utilisation : un Escape game}
\label{\detokenize{projets/boite-fermee-exemple-escape::doc}}\label{\detokenize{projets/boite-fermee-exemple-escape:projet-projetboite-exemple-d-utilisation-un-escape-game}}
Nous avons utilisé le projet {\hyperref[\detokenize{projets/boite-fermee:projetboite}]{\sphinxcrossref{\DUrole{std,std-ref}{Boîte fermée}}}} pour un escape
game proposé en stage.
\begin{itemize}
\item {} 
diaporama d’acceuil : \sphinxurl{http://url.univ-irem.fr/boite}

\item {} 
page de formation : \sphinxurl{http://url.univ-irem.fr/algo1718-boite}

\end{itemize}


\subsection{Réalisation}
\label{\detokenize{projets/boite-fermee:realisation}}

\subsubsection{Projet Boîte fermée - Fabriquer}
\label{\detokenize{projets/boite-fermee-fabriquer:projet-projetboite-fabriquer}}\label{\detokenize{projets/boite-fermee-fabriquer::doc}}
Nous détaillons ici comment fabriquer et assembler
le matériel nécessaire à la réalisation du projet
{\hyperref[\detokenize{projets/boite-fermee:projetboite}]{\sphinxcrossref{\DUrole{std,std-ref}{Boîte fermée}}}}.

\begin{sphinxadmonition}{note}{\label{projets/boite-fermee-fabriquer:index-0}À faire:}
tout faire.
\end{sphinxadmonition}


\subsubsection{Projet Boîte fermée - Coder}
\label{\detokenize{projets/boite-fermee-coder:projet-projetboite-coder}}\label{\detokenize{projets/boite-fermee-coder::doc}}
Nous détaillons ici le code nécessaire à la réalisation
du projet {\hyperref[\detokenize{projets/boite-fermee:projetboite}]{\sphinxcrossref{\DUrole{std,std-ref}{Boîte fermée}}}}.

\noindent{\hspace*{\fill}\sphinxincludegraphics[width=0.650\linewidth]{{boite-fermee}.png}\hspace*{\fill}}


\section{Planche de Galton}
\label{\detokenize{projets/galton:planche-de-galton}}\label{\detokenize{projets/galton::doc}}\label{\detokenize{projets/galton:projetgalton}}

\subsection{Description}
\label{\detokenize{projets/galton:description}}
\begin{sphinxadmonition}{note}{\label{projets/galton:index-0}À faire:}
capture d’écran / gif animée
\end{sphinxadmonition}


\subsubsection{Exemple(s) d’utilisation}
\label{\detokenize{projets/galton:exemple-s-d-utilisation}}

\subsection{Réalisation}
\label{\detokenize{projets/galton:realisation}}

\subsubsection{Projet Planche de Galton - Fabriquer}
\label{\detokenize{projets/galton-fabriquer:projet-projetgalton-fabriquer}}\label{\detokenize{projets/galton-fabriquer::doc}}
Nous détaillons ici comment fabriquer et assembler
le matériel nécessaire à la réalisation du projet
{\hyperref[\detokenize{projets/galton:projetgalton}]{\sphinxcrossref{\DUrole{std,std-ref}{Planche de Galton}}}}.

\begin{sphinxadmonition}{note}{\label{projets/galton-fabriquer:index-0}À faire:}
tout faire.
\end{sphinxadmonition}


\subsubsection{Projet Planche de Galton - Coder}
\label{\detokenize{projets/galton-coder:projet-projetgalton-coder}}\label{\detokenize{projets/galton-coder::doc}}
Nous détaillons ici le code nécessaire à la réalisation
du projet {\hyperref[\detokenize{projets/galton:projetgalton}]{\sphinxcrossref{\DUrole{std,std-ref}{Planche de Galton}}}}.

\fvset{hllines={, ,}}%
\begin{sphinxVerbatim}[commandchars=\\\{\}]
\PYG{k+kn}{from} \PYG{n+nn}{microbit} \PYG{k}{import} \PYG{o}{*}
\PYG{k+kn}{from} \PYG{n+nn}{random} \PYG{k}{import} \PYG{n}{random}\PYG{p}{,} \PYG{n}{seed}

\PYG{n}{seed}\PYG{p}{(}\PYG{l+m+mi}{300}\PYG{p}{)}               \PYG{c+c1}{\PYGZsh{} la graine de hasard ???}
\PYG{n}{n} \PYG{o}{=} \PYG{p}{[}\PYG{l+m+mi}{0}\PYG{p}{,} \PYG{l+m+mi}{0}\PYG{p}{,} \PYG{l+m+mi}{0}\PYG{p}{,} \PYG{l+m+mi}{0}\PYG{p}{,} \PYG{l+m+mi}{0}\PYG{p}{]}     \PYG{c+c1}{\PYGZsh{} le tableau contenant les compteurs}


\PYG{k}{def} \PYG{n+nf}{aff}\PYG{p}{(}\PYG{n}{n}\PYG{p}{,} \PYG{n}{m}\PYG{p}{)}\PYG{p}{:}          \PYG{c+c1}{\PYGZsh{} la fonction affichant le graph}
    \PYG{n}{q} \PYG{o}{=} \PYG{n}{n} \PYG{o}{/}\PYG{o}{/} \PYG{l+m+mi}{9}          \PYG{c+c1}{\PYGZsh{} nombre de led eclaire totalement}
    \PYG{n}{r} \PYG{o}{=} \PYG{n}{n} \PYG{o}{\PYGZpc{}} \PYG{l+m+mi}{9}           \PYG{c+c1}{\PYGZsh{} portion de la derniere led eclaire}
    \PYG{k}{for} \PYG{n}{i} \PYG{o+ow}{in} \PYG{n+nb}{range}\PYG{p}{(}\PYG{l+m+mi}{0}\PYG{p}{,} \PYG{n}{q}\PYG{p}{)}\PYG{p}{:}
        \PYG{n}{display}\PYG{o}{.}\PYG{n}{set\PYGZus{}pixel}\PYG{p}{(}\PYG{n}{m}\PYG{p}{,} \PYG{l+m+mi}{4}\PYG{o}{\PYGZhy{}}\PYG{n}{i}\PYG{p}{,} \PYG{l+m+mi}{9}\PYG{p}{)}
    \PYG{n}{display}\PYG{o}{.}\PYG{n}{set\PYGZus{}pixel}\PYG{p}{(}\PYG{n}{m}\PYG{p}{,} \PYG{l+m+mi}{4}\PYG{o}{\PYGZhy{}}\PYG{n}{q}\PYG{p}{,} \PYG{n}{r}\PYG{p}{)}


\PYG{k}{def} \PYG{n+nf}{chute}\PYG{p}{(}\PYG{n}{t}\PYG{p}{)}\PYG{p}{:}                \PYG{c+c1}{\PYGZsh{} fonction affichant la chute}
    \PYG{n}{display}\PYG{o}{.}\PYG{n}{clear}\PYG{p}{(}\PYG{p}{)}
    \PYG{n}{y}\PYG{p}{,} \PYG{n}{x} \PYG{o}{=} \PYG{l+m+mi}{0}\PYG{p}{,} \PYG{l+m+mi}{0}
    \PYG{n}{display}\PYG{o}{.}\PYG{n}{set\PYGZus{}pixel}\PYG{p}{(}\PYG{n}{x}\PYG{p}{,} \PYG{n}{y}\PYG{p}{,} \PYG{l+m+mi}{9}\PYG{p}{)}
    \PYG{n}{sleep}\PYG{p}{(}\PYG{n}{t}\PYG{p}{)}
    \PYG{k}{while} \PYG{n}{y} \PYG{o}{\PYGZlt{}} \PYG{l+m+mi}{4}\PYG{p}{:}
        \PYG{n}{display}\PYG{o}{.}\PYG{n}{clear}\PYG{p}{(}\PYG{p}{)}
        \PYG{k}{if} \PYG{n+nb}{round}\PYG{p}{(}\PYG{n}{random}\PYG{p}{(}\PYG{p}{)}\PYG{p}{)}\PYG{p}{:}      \PYG{c+c1}{\PYGZsh{} si arrondi de alea est vrai (différent de 0)}
            \PYG{n}{y} \PYG{o}{=} \PYG{n}{y} \PYG{o}{+} \PYG{l+m+mi}{1}            \PYG{c+c1}{\PYGZsh{} on augmente y de 1}
        \PYG{k}{else}\PYG{p}{:}
            \PYG{n}{x} \PYG{o}{=} \PYG{n}{x} \PYG{o}{+} \PYG{l+m+mi}{1}
            \PYG{n}{y} \PYG{o}{=} \PYG{n}{y} \PYG{o}{+} \PYG{l+m+mi}{1}
        \PYG{n}{display}\PYG{o}{.}\PYG{n}{set\PYGZus{}pixel}\PYG{p}{(}\PYG{n}{x}\PYG{p}{,} \PYG{n}{y}\PYG{p}{,} \PYG{l+m+mi}{9}\PYG{p}{)}
        \PYG{n}{sleep}\PYG{p}{(}\PYG{n}{t}\PYG{p}{)}
    \PYG{n}{n}\PYG{p}{[}\PYG{n}{x}\PYG{p}{]} \PYG{o}{=} \PYG{n}{n}\PYG{p}{[}\PYG{n}{x}\PYG{p}{]}\PYG{o}{+}\PYG{l+m+mi}{1}        \PYG{c+c1}{\PYGZsh{} incrementation du compteur de la position x}
    \PYG{n}{display}\PYG{o}{.}\PYG{n}{set\PYGZus{}pixel}\PYG{p}{(}\PYG{n}{x}\PYG{p}{,} \PYG{n}{y}\PYG{p}{,} \PYG{l+m+mi}{1}\PYG{p}{)}


\PYG{k}{while} \PYG{k+kc}{True}\PYG{p}{:}
    \PYG{k}{if} \PYG{n}{button\PYGZus{}a}\PYG{o}{.}\PYG{n}{is\PYGZus{}pressed}\PYG{p}{(}\PYG{p}{)}\PYG{p}{:}
        \PYG{n}{chute}\PYG{p}{(}\PYG{l+m+mi}{500}\PYG{p}{)}

    \PYG{k}{elif} \PYG{n}{button\PYGZus{}b}\PYG{o}{.}\PYG{n}{get\PYGZus{}presses}\PYG{p}{(}\PYG{p}{)}\PYG{p}{:}
        \PYG{n}{n} \PYG{o}{=} \PYG{p}{[}\PYG{l+m+mi}{0}\PYG{p}{,} \PYG{l+m+mi}{0}\PYG{p}{,} \PYG{l+m+mi}{0}\PYG{p}{,} \PYG{l+m+mi}{0}\PYG{p}{,} \PYG{l+m+mi}{0}\PYG{p}{]}
        \PYG{k}{for} \PYG{n}{k} \PYG{o+ow}{in} \PYG{n+nb}{range}\PYG{p}{(}\PYG{l+m+mi}{80}\PYG{p}{)}\PYG{p}{:}
            \PYG{n}{chute}\PYG{p}{(}\PYG{n+nb}{round}\PYG{p}{(}\PYG{l+m+mi}{500} \PYG{o}{/} \PYG{p}{(}\PYG{l+m+mf}{1.05}\PYG{o}{*}\PYG{o}{*}\PYG{n}{k}\PYG{p}{)}\PYG{p}{)}\PYG{p}{)}
            \PYG{k}{for} \PYG{n}{j} \PYG{o+ow}{in} \PYG{n+nb}{range}\PYG{p}{(}\PYG{l+m+mi}{5}\PYG{p}{)}\PYG{p}{:}
                \PYG{n}{aff}\PYG{p}{(}\PYG{n}{n}\PYG{p}{[}\PYG{n}{j}\PYG{p}{]}\PYG{p}{,} \PYG{n}{j}\PYG{p}{)}
            \PYG{n}{sleep}\PYG{p}{(}\PYG{l+m+mi}{200}\PYG{p}{)}
        \PYG{n+nb}{print}\PYG{p}{(}\PYG{n}{n}\PYG{p}{)}
\end{sphinxVerbatim}


\chapter{Index et page de recherche}
\label{\detokenize{index:index-et-page-de-recherche}}\begin{itemize}
\item {} 
\DUrole{xref,std,std-ref}{genindex}

\item {} 
\DUrole{xref,std,std-ref}{search}

\end{itemize}


\bigskip\hrule\bigskip

\sphinxhref{http://microbit.readthedocs.io/fr/latest/?badge=latest}{}


\renewcommand{\indexname}{Index}
\printindex
\end{document}