%% Generated by Sphinx.
\def\sphinxdocclass{report}
\documentclass[letterpaper,10pt,french]{sphinxmanual}
\ifdefined\pdfpxdimen
   \let\sphinxpxdimen\pdfpxdimen\else\newdimen\sphinxpxdimen
\fi \sphinxpxdimen=.75bp\relax

\PassOptionsToPackage{warn}{textcomp}
\usepackage[utf8]{inputenc}
\ifdefined\DeclareUnicodeCharacter
 \ifdefined\DeclareUnicodeCharacterAsOptional
  \DeclareUnicodeCharacter{"00A0}{\nobreakspace}
  \DeclareUnicodeCharacter{"2500}{\sphinxunichar{2500}}
  \DeclareUnicodeCharacter{"2502}{\sphinxunichar{2502}}
  \DeclareUnicodeCharacter{"2514}{\sphinxunichar{2514}}
  \DeclareUnicodeCharacter{"251C}{\sphinxunichar{251C}}
  \DeclareUnicodeCharacter{"2572}{\textbackslash}
 \else
  \DeclareUnicodeCharacter{00A0}{\nobreakspace}
  \DeclareUnicodeCharacter{2500}{\sphinxunichar{2500}}
  \DeclareUnicodeCharacter{2502}{\sphinxunichar{2502}}
  \DeclareUnicodeCharacter{2514}{\sphinxunichar{2514}}
  \DeclareUnicodeCharacter{251C}{\sphinxunichar{251C}}
  \DeclareUnicodeCharacter{2572}{\textbackslash}
 \fi
\fi
\usepackage{cmap}
\usepackage[T1]{fontenc}
\usepackage{amsmath,amssymb,amstext}
\usepackage[french]{babel}
\usepackage{times}
\usepackage[Sonny]{fncychap}
\ChNameVar{\Large\normalfont\sffamily}
\ChTitleVar{\Large\normalfont\sffamily}
\usepackage{sphinx}

\usepackage{geometry}

% Include hyperref last.
\usepackage{hyperref}
% Fix anchor placement for figures with captions.
\usepackage{hypcap}% it must be loaded after hyperref.
% Set up styles of URL: it should be placed after hyperref.
\urlstyle{same}
\addto\captionsfrench{\renewcommand{\contentsname}{Projets}}

\addto\captionsfrench{\renewcommand{\figurename}{Fig.}}
\addto\captionsfrench{\renewcommand{\tablename}{Tableau}}
\addto\captionsfrench{\renewcommand{\literalblockname}{Code source}}

\addto\captionsfrench{\renewcommand{\literalblockcontinuedname}{suite de la page précédente}}
\addto\captionsfrench{\renewcommand{\literalblockcontinuesname}{suite sur la page suivante}}

\addto\extrasfrench{\def\pageautorefname{page}}

\setcounter{tocdepth}{1}



\title{Ressources pour la carte Micro:bit}
\date{mai 09, 2018}
\release{1}
\author{IREM Marseille}
\newcommand{\sphinxlogo}{\vbox{}}
\renewcommand{\releasename}{Version}
\makeindex

\begin{document}

\maketitle
\sphinxtableofcontents
\phantomsection\label{\detokenize{index::doc}}


par le groupe InEFLP de l’IREM de Marseille


\chapter{Projets à réaliser}
\label{\detokenize{index:documentation-micro-bit}}\label{\detokenize{index:projets-a-realiser}}

\section{Températures}
\label{\detokenize{projets/temperature:projettemp}}\label{\detokenize{projets/temperature::doc}}\label{\detokenize{projets/temperature:temperatures}}

\subsection{Description}
\label{\detokenize{projets/temperature:description}}
\begin{sphinxadmonition}{note}{\label{projets/temperature:index-0}À faire:}
capture d’écran / gif animée.
\end{sphinxadmonition}


\subsubsection{Exemple(s) d’utilisation}
\label{\detokenize{projets/temperature:exemple-s-d-utilisation}}

\paragraph{Escape game}
\label{\detokenize{projets/temperature-exemple-escape:projettempescape}}\label{\detokenize{projets/temperature-exemple-escape:escape-game}}\label{\detokenize{projets/temperature-exemple-escape::doc}}
Nous avons utilisé le projet {\hyperref[\detokenize{projets/temperature:projettemp}]{\sphinxcrossref{\DUrole{std,std-ref}{Températures}}}} pour un escape game proposé en stage.
\begin{itemize}
\item {} 
diaporama d’acceuil : \sphinxurl{http://url.univ-irem.fr/temp}

\item {} 
page de formation : \sphinxurl{http://url.univ-irem.fr/algo1718-temp}

\end{itemize}


\subsection{Réalisation}
\label{\detokenize{projets/temperature:realisation}}

\subsubsection{Fabriquer}
\label{\detokenize{projets/temperature-fabriquer:projettempfabriquer}}\label{\detokenize{projets/temperature-fabriquer::doc}}\label{\detokenize{projets/temperature-fabriquer:fabriquer}}
Nous détaillons ici comment fabriquer et assembler
le matériel nécessaire à la réalisation du projet
{\hyperref[\detokenize{projets/temperature:projettemp}]{\sphinxcrossref{\DUrole{std,std-ref}{Températures}}}}.

\begin{sphinxadmonition}{note}{\label{projets/temperature-fabriquer:index-0}À faire:}
tout faire.
\end{sphinxadmonition}


\subsubsection{Coder}
\label{\detokenize{projets/temperature-coder:projettempcoder}}\label{\detokenize{projets/temperature-coder:coder}}\label{\detokenize{projets/temperature-coder::doc}}
Ce projet a été programmé en micropython.

Nous détaillons ici le code nécessaire à la réalisation
du projet {\hyperref[\detokenize{projets/temperature:projettemp}]{\sphinxcrossref{\DUrole{std,std-ref}{Températures}}}}.

\index{micropython}\index{projet}\ignorespaces 

\paragraph{Le code, étape par étape}
\label{\detokenize{projets/temperature-coder:le-code-etape-par-etape}}\label{\detokenize{projets/temperature-coder:index-0}}
Incluons la bibliothèque Micro:bit

\fvset{hllines={, ,}}%
\begin{sphinxVerbatim}[commandchars=\\\{\}]
\PYG{k+kn}{from} \PYG{n+nn}{microbit} \PYG{k}{import} \PYG{o}{*}
\end{sphinxVerbatim}


\paragraph{Le code final}
\label{\detokenize{projets/temperature-coder:le-code-final}}
\fvset{hllines={, ,}}%
\begin{sphinxVerbatim}[commandchars=\\\{\},numbers=left,firstnumber=1,stepnumber=1]
\PYG{c+c1}{\PYGZsh{} \PYGZhy{}*\PYGZhy{} coding: utf\PYGZhy{}8\PYGZhy{}*\PYGZhy{}\PYGZsh{} Encoding cookie added by Mu Editor}
\PYG{k+kn}{from} \PYG{n+nn}{microbit} \PYG{k}{import} \PYG{o}{*}

\PYG{c+c1}{\PYGZsh{} définir mes images persos}
\PYG{c+c1}{\PYGZsh{} pour les lignes qui se colorent}
\PYG{n}{image1} \PYG{o}{=} \PYG{n}{Image}\PYG{p}{(}
    \PYG{l+s+s1}{\PYGZsq{}}\PYG{l+s+s1}{00000:}\PYG{l+s+s1}{\PYGZsq{}}
    \PYG{l+s+s1}{\PYGZsq{}}\PYG{l+s+s1}{00000:}\PYG{l+s+s1}{\PYGZsq{}}
    \PYG{l+s+s1}{\PYGZsq{}}\PYG{l+s+s1}{00000:}\PYG{l+s+s1}{\PYGZsq{}}
    \PYG{l+s+s1}{\PYGZsq{}}\PYG{l+s+s1}{00000:}\PYG{l+s+s1}{\PYGZsq{}}
    \PYG{l+s+s1}{\PYGZsq{}}\PYG{l+s+s1}{99999}\PYG{l+s+s1}{\PYGZsq{}}\PYG{p}{)}    
\PYG{n}{image2} \PYG{o}{=} \PYG{n}{Image}\PYG{p}{(}
    \PYG{l+s+s1}{\PYGZsq{}}\PYG{l+s+s1}{00000:}\PYG{l+s+s1}{\PYGZsq{}}
    \PYG{l+s+s1}{\PYGZsq{}}\PYG{l+s+s1}{00000:}\PYG{l+s+s1}{\PYGZsq{}}
    \PYG{l+s+s1}{\PYGZsq{}}\PYG{l+s+s1}{00000:}\PYG{l+s+s1}{\PYGZsq{}}
    \PYG{l+s+s1}{\PYGZsq{}}\PYG{l+s+s1}{99999:}\PYG{l+s+s1}{\PYGZsq{}}
    \PYG{l+s+s1}{\PYGZsq{}}\PYG{l+s+s1}{77777}\PYG{l+s+s1}{\PYGZsq{}}\PYG{p}{)}
\PYG{n}{image3} \PYG{o}{=} \PYG{n}{Image}\PYG{p}{(}
    \PYG{l+s+s1}{\PYGZsq{}}\PYG{l+s+s1}{00000:}\PYG{l+s+s1}{\PYGZsq{}}
    \PYG{l+s+s1}{\PYGZsq{}}\PYG{l+s+s1}{00000:}\PYG{l+s+s1}{\PYGZsq{}}
    \PYG{l+s+s1}{\PYGZsq{}}\PYG{l+s+s1}{99999:}\PYG{l+s+s1}{\PYGZsq{}}
    \PYG{l+s+s1}{\PYGZsq{}}\PYG{l+s+s1}{77777:}\PYG{l+s+s1}{\PYGZsq{}}
    \PYG{l+s+s1}{\PYGZsq{}}\PYG{l+s+s1}{77777}\PYG{l+s+s1}{\PYGZsq{}}\PYG{p}{)}
\PYG{n}{image4} \PYG{o}{=} \PYG{n}{Image}\PYG{p}{(}
    \PYG{l+s+s1}{\PYGZsq{}}\PYG{l+s+s1}{00000:}\PYG{l+s+s1}{\PYGZsq{}}
    \PYG{l+s+s1}{\PYGZsq{}}\PYG{l+s+s1}{99999:}\PYG{l+s+s1}{\PYGZsq{}}
    \PYG{l+s+s1}{\PYGZsq{}}\PYG{l+s+s1}{77777:}\PYG{l+s+s1}{\PYGZsq{}}
    \PYG{l+s+s1}{\PYGZsq{}}\PYG{l+s+s1}{77777:}\PYG{l+s+s1}{\PYGZsq{}}
    \PYG{l+s+s1}{\PYGZsq{}}\PYG{l+s+s1}{77777}\PYG{l+s+s1}{\PYGZsq{}}\PYG{p}{)}
\PYG{n}{image5} \PYG{o}{=} \PYG{n}{Image}\PYG{p}{(}
    \PYG{l+s+s1}{\PYGZsq{}}\PYG{l+s+s1}{99999:}\PYG{l+s+s1}{\PYGZsq{}}
    \PYG{l+s+s1}{\PYGZsq{}}\PYG{l+s+s1}{77777:}\PYG{l+s+s1}{\PYGZsq{}}
    \PYG{l+s+s1}{\PYGZsq{}}\PYG{l+s+s1}{77777:}\PYG{l+s+s1}{\PYGZsq{}}
    \PYG{l+s+s1}{\PYGZsq{}}\PYG{l+s+s1}{77777:}\PYG{l+s+s1}{\PYGZsq{}}
    \PYG{l+s+s1}{\PYGZsq{}}\PYG{l+s+s1}{77777}\PYG{l+s+s1}{\PYGZsq{}}\PYG{p}{)}

\PYG{c+c1}{\PYGZsh{} booléen pour savoir si l\PYGZsq{}énigme est réussie}
\PYG{n}{victoire} \PYG{o}{=} \PYG{k+kc}{False}

\PYG{c+c1}{\PYGZsh{} à faire toujours et toujours…}
\PYG{k}{while} \PYG{k+kc}{True}\PYG{p}{:}
    \PYG{c+c1}{\PYGZsh{} si  l\PYGZsq{}énigme n\PYGZsq{}a pas été résolue}
    \PYG{k}{if} \PYG{o+ow}{not} \PYG{n}{victoire}\PYG{p}{:}
        \PYG{c+c1}{\PYGZsh{} lire la température (en °C)}
        \PYG{n}{temp} \PYG{o}{=} \PYG{n}{temperature}\PYG{p}{(}\PYG{p}{)}
        \PYG{c+c1}{\PYGZsh{} affichage des images en fonction}
        \PYG{c+c1}{\PYGZsh{} de temp}
        \PYG{k}{if} \PYG{n}{temp} \PYG{o}{\PYGZlt{}} \PYG{l+m+mi}{29}\PYG{p}{:}
            \PYG{n}{display}\PYG{o}{.}\PYG{n}{clear}\PYG{p}{(}\PYG{p}{)}
        \PYG{k}{elif} \PYG{l+m+mi}{29} \PYG{o}{\PYGZlt{}}\PYG{o}{=} \PYG{n}{temp} \PYG{o}{\PYGZlt{}} \PYG{l+m+mi}{30}\PYG{p}{:}
            \PYG{n}{display}\PYG{o}{.}\PYG{n}{show}\PYG{p}{(}\PYG{n}{image1}\PYG{p}{)}
        \PYG{k}{elif} \PYG{l+m+mi}{30} \PYG{o}{\PYGZlt{}}\PYG{o}{=} \PYG{n}{temp} \PYG{o}{\PYGZlt{}} \PYG{l+m+mi}{31}\PYG{p}{:}
            \PYG{n}{display}\PYG{o}{.}\PYG{n}{show}\PYG{p}{(}\PYG{n}{image2}\PYG{p}{)}
        \PYG{k}{elif} \PYG{l+m+mi}{31} \PYG{o}{\PYGZlt{}}\PYG{o}{=} \PYG{n}{temp} \PYG{o}{\PYGZlt{}} \PYG{l+m+mi}{32}\PYG{p}{:}
            \PYG{n}{display}\PYG{o}{.}\PYG{n}{show}\PYG{p}{(}\PYG{n}{image3}\PYG{p}{)}
        \PYG{k}{elif} \PYG{l+m+mi}{32} \PYG{o}{\PYGZlt{}}\PYG{o}{=} \PYG{n}{temp} \PYG{o}{\PYGZlt{}} \PYG{l+m+mi}{33}\PYG{p}{:}
            \PYG{n}{display}\PYG{o}{.}\PYG{n}{show}\PYG{p}{(}\PYG{n}{image4}\PYG{p}{)}
        \PYG{k}{elif} \PYG{l+m+mi}{33} \PYG{o}{\PYGZlt{}}\PYG{o}{=} \PYG{n}{temp} \PYG{o}{\PYGZlt{}} \PYG{l+m+mi}{34}\PYG{p}{:}
            \PYG{n}{display}\PYG{o}{.}\PYG{n}{show}\PYG{p}{(}\PYG{n}{image5}\PYG{p}{)}
        \PYG{c+c1}{\PYGZsh{} victoire !}
        \PYG{k}{elif} \PYG{l+m+mi}{34} \PYG{o}{\PYGZlt{}}\PYG{o}{=} \PYG{n}{temp}\PYG{p}{:}
            \PYG{n}{victoire} \PYG{o}{=} \PYG{k+kc}{True}
            \PYG{c+c1}{\PYGZsh{} petite animation}
            \PYG{k}{for} \PYG{n}{i} \PYG{o+ow}{in} \PYG{n+nb}{range}\PYG{p}{(}\PYG{l+m+mi}{2}\PYG{p}{)}\PYG{p}{:}
                \PYG{n}{display}\PYG{o}{.}\PYG{n}{show}\PYG{p}{(}\PYG{n}{Image}\PYG{o}{.}\PYG{n}{SQUARE\PYGZus{}SMALL}\PYG{p}{)}
                \PYG{n}{sleep}\PYG{p}{(}\PYG{l+m+mi}{100}\PYG{p}{)}
                \PYG{n}{display}\PYG{o}{.}\PYG{n}{show}\PYG{p}{(}\PYG{n}{Image}\PYG{o}{.}\PYG{n}{SQUARE}\PYG{p}{)}
                \PYG{n}{sleep}\PYG{p}{(}\PYG{l+m+mi}{100}\PYG{p}{)}
        \PYG{n}{sleep}\PYG{p}{(}\PYG{l+m+mi}{500}\PYG{p}{)}
    
    \PYG{c+c1}{\PYGZsh{} si l\PYGZsq{}énigme est résolue}
    \PYG{k}{if} \PYG{n}{victoire}\PYG{p}{:}
        \PYG{c+c1}{\PYGZsh{} petite image joyeuse}
        \PYG{n}{display}\PYG{o}{.}\PYG{n}{show}\PYG{p}{(}\PYG{n}{Image}\PYG{o}{.}\PYG{n}{HAPPY}\PYG{p}{)}
        \PYG{n}{sleep}\PYG{p}{(}\PYG{l+m+mi}{500}\PYG{p}{)}
        \PYG{c+c1}{\PYGZsh{} code secret à afficher…}
        \PYG{n}{display}\PYG{o}{.}\PYG{n}{scroll}\PYG{p}{(}\PYG{l+s+s2}{\PYGZdq{}}\PYG{l+s+s2}{XXXXXX}\PYG{l+s+s2}{\PYGZdq{}}\PYG{p}{)}
    
    \PYG{c+c1}{\PYGZsh{} utiliser le bouton A pour réinitialiser}
    \PYG{k}{if} \PYG{n}{button\PYGZus{}a}\PYG{o}{.}\PYG{n}{is\PYGZus{}pressed}\PYG{p}{(}\PYG{p}{)}\PYG{p}{:}
        \PYG{n}{victoire} \PYG{o}{=} \PYG{k+kc}{False}
\end{sphinxVerbatim}


\section{Coffre fort}
\label{\detokenize{projets/coffre::doc}}\label{\detokenize{projets/coffre:coffre-fort}}\label{\detokenize{projets/coffre:projetcoffre}}

\subsection{Description}
\label{\detokenize{projets/coffre:description}}
\begin{sphinxadmonition}{note}{\label{projets/coffre:index-0}À faire:}
capture d’écran / gif animée
\end{sphinxadmonition}


\subsubsection{Exemple(s) d’utilisation}
\label{\detokenize{projets/coffre:exemple-s-d-utilisation}}

\paragraph{Escape game}
\label{\detokenize{projets/coffre-exemple-escape:escape-game}}\label{\detokenize{projets/coffre-exemple-escape::doc}}
Nous avons utilisé le projet {\hyperref[\detokenize{projets/coffre:projetcoffre}]{\sphinxcrossref{\DUrole{std,std-ref}{Coffre fort}}}} pour un escape
game proposé en stage.
\begin{itemize}
\item {} 
diaporama d’acceuil : \sphinxurl{http://url.univ-irem.fr/coffre}

\item {} 
page de formation : \sphinxurl{http://url.univ-irem.fr/algo1718-coffre}

\end{itemize}


\subsection{Réalisation}
\label{\detokenize{projets/coffre:realisation}}

\subsubsection{Fabriquer}
\label{\detokenize{projets/coffre-fabriquer::doc}}\label{\detokenize{projets/coffre-fabriquer:fabriquer}}
Nous détaillons ici comment fabriquer et assembler
le matériel nécessaire à la réalisation du projet
{\hyperref[\detokenize{projets/coffre:projetcoffre}]{\sphinxcrossref{\DUrole{std,std-ref}{Coffre fort}}}}.

\begin{sphinxadmonition}{note}{\label{projets/coffre-fabriquer:index-0}À faire:}
tout faire.
\end{sphinxadmonition}

\index{projet}\index{micropython}\index{python|see{micropython}}\ignorespaces 

\subsubsection{Coder}
\label{\detokenize{projets/coffre-coder:index-0}}\label{\detokenize{projets/coffre-coder:coder}}\label{\detokenize{projets/coffre-coder::doc}}
Nous détaillons ici le code nécessaire à la réalisation
du projet {\hyperref[\detokenize{projets/coffre:projetcoffre}]{\sphinxcrossref{\DUrole{std,std-ref}{Coffre fort}}}}.

\begin{sphinxadmonition}{note}{\label{projets/coffre-coder:index-1}À faire:}
tout à faire!
\end{sphinxadmonition}


\section{Boîte fermée}
\label{\detokenize{projets/boite-fermee:boite-fermee}}\label{\detokenize{projets/boite-fermee::doc}}\label{\detokenize{projets/boite-fermee:projetboite}}

\subsection{Description}
\label{\detokenize{projets/boite-fermee:description}}
\begin{sphinxadmonition}{note}{\label{projets/boite-fermee:index-0}À faire:}
capture d’écran / gif animée
\end{sphinxadmonition}


\subsubsection{Exemple(s) d’utilisation}
\label{\detokenize{projets/boite-fermee:exemple-s-d-utilisation}}

\paragraph{Escape game}
\label{\detokenize{projets/boite-fermee-exemple-escape:escape-game}}\label{\detokenize{projets/boite-fermee-exemple-escape::doc}}
Nous avons utilisé le projet {\hyperref[\detokenize{projets/boite-fermee:projetboite}]{\sphinxcrossref{\DUrole{std,std-ref}{Boîte fermée}}}} pour un escape
game proposé en stage.
\begin{itemize}
\item {} 
diaporama d’acceuil : \sphinxurl{http://url.univ-irem.fr/boite}

\item {} 
page de formation : \sphinxurl{http://url.univ-irem.fr/algo1718-boite}

\end{itemize}


\subsection{Réalisation}
\label{\detokenize{projets/boite-fermee:realisation}}

\subsubsection{Fabriquer}
\label{\detokenize{projets/boite-fermee-fabriquer::doc}}\label{\detokenize{projets/boite-fermee-fabriquer:fabriquer}}
Nous détaillons ici comment fabriquer et assembler
le matériel nécessaire à la réalisation du projet
{\hyperref[\detokenize{projets/boite-fermee:projetboite}]{\sphinxcrossref{\DUrole{std,std-ref}{Boîte fermée}}}}.

\begin{sphinxadmonition}{note}{\label{projets/boite-fermee-fabriquer:index-0}À faire:}
tout faire.
\end{sphinxadmonition}


\subsubsection{Coder}
\label{\detokenize{projets/boite-fermee-coder:coder}}\label{\detokenize{projets/boite-fermee-coder::doc}}
Nous détaillons ici le code nécessaire à la réalisation
du projet {\hyperref[\detokenize{projets/boite-fermee:projetboite}]{\sphinxcrossref{\DUrole{std,std-ref}{Boîte fermée}}}}.

\noindent{\hspace*{\fill}\sphinxincludegraphics[width=0.650\linewidth]{{boite-fermee}.png}\hspace*{\fill}}


\section{Planche de Galton}
\label{\detokenize{projets/galton:planche-de-galton}}\label{\detokenize{projets/galton::doc}}\label{\detokenize{projets/galton:projetgalton}}

\subsection{Description}
\label{\detokenize{projets/galton:description}}
\begin{sphinxadmonition}{note}{\label{projets/galton:index-0}À faire:}
capture d’écran / gif animée
\end{sphinxadmonition}


\subsubsection{Exemple(s) d’utilisation}
\label{\detokenize{projets/galton:exemple-s-d-utilisation}}

\subsection{Réalisation}
\label{\detokenize{projets/galton:realisation}}

\subsubsection{Fabriquer}
\label{\detokenize{projets/galton-fabriquer::doc}}\label{\detokenize{projets/galton-fabriquer:fabriquer}}
Nous détaillons ici comment fabriquer et assembler
le matériel nécessaire à la réalisation du projet
{\hyperref[\detokenize{projets/galton:projetgalton}]{\sphinxcrossref{\DUrole{std,std-ref}{Planche de Galton}}}}.

\begin{sphinxadmonition}{note}{\label{projets/galton-fabriquer:index-0}À faire:}
tout faire.
\end{sphinxadmonition}


\subsubsection{Coder}
\label{\detokenize{projets/galton-coder:coder}}\label{\detokenize{projets/galton-coder::doc}}
\index{micropython}\index{galton}\index{projet}\ignorespaces 
Nous détaillons ici le code nécessaire à la réalisation
du projet {\hyperref[\detokenize{projets/galton:projetgalton}]{\sphinxcrossref{\DUrole{std,std-ref}{Planche de Galton}}}}.

\fvset{hllines={, ,}}%
\begin{sphinxVerbatim}[commandchars=\\\{\}]
\PYG{k+kn}{from} \PYG{n+nn}{microbit} \PYG{k}{import} \PYG{o}{*}
\PYG{k+kn}{from} \PYG{n+nn}{random} \PYG{k}{import} \PYG{n}{random}\PYG{p}{,} \PYG{n}{seed}

\PYG{n}{seed}\PYG{p}{(}\PYG{l+m+mi}{300}\PYG{p}{)}               \PYG{c+c1}{\PYGZsh{} la graine de hasard ???}
\PYG{n}{n} \PYG{o}{=} \PYG{p}{[}\PYG{l+m+mi}{0}\PYG{p}{,} \PYG{l+m+mi}{0}\PYG{p}{,} \PYG{l+m+mi}{0}\PYG{p}{,} \PYG{l+m+mi}{0}\PYG{p}{,} \PYG{l+m+mi}{0}\PYG{p}{]}     \PYG{c+c1}{\PYGZsh{} le tableau contenant les compteurs}


\PYG{k}{def} \PYG{n+nf}{aff}\PYG{p}{(}\PYG{n}{n}\PYG{p}{,} \PYG{n}{m}\PYG{p}{)}\PYG{p}{:}          \PYG{c+c1}{\PYGZsh{} la fonction affichant le graph}
    \PYG{n}{q} \PYG{o}{=} \PYG{n}{n} \PYG{o}{/}\PYG{o}{/} \PYG{l+m+mi}{9}          \PYG{c+c1}{\PYGZsh{} nombre de led eclaire totalement}
    \PYG{n}{r} \PYG{o}{=} \PYG{n}{n} \PYG{o}{\PYGZpc{}} \PYG{l+m+mi}{9}           \PYG{c+c1}{\PYGZsh{} portion de la derniere led eclaire}
    \PYG{k}{for} \PYG{n}{i} \PYG{o+ow}{in} \PYG{n+nb}{range}\PYG{p}{(}\PYG{l+m+mi}{0}\PYG{p}{,} \PYG{n}{q}\PYG{p}{)}\PYG{p}{:}
        \PYG{n}{display}\PYG{o}{.}\PYG{n}{set\PYGZus{}pixel}\PYG{p}{(}\PYG{n}{m}\PYG{p}{,} \PYG{l+m+mi}{4}\PYG{o}{\PYGZhy{}}\PYG{n}{i}\PYG{p}{,} \PYG{l+m+mi}{9}\PYG{p}{)}
    \PYG{n}{display}\PYG{o}{.}\PYG{n}{set\PYGZus{}pixel}\PYG{p}{(}\PYG{n}{m}\PYG{p}{,} \PYG{l+m+mi}{4}\PYG{o}{\PYGZhy{}}\PYG{n}{q}\PYG{p}{,} \PYG{n}{r}\PYG{p}{)}


\PYG{k}{def} \PYG{n+nf}{chute}\PYG{p}{(}\PYG{n}{t}\PYG{p}{)}\PYG{p}{:}                \PYG{c+c1}{\PYGZsh{} fonction affichant la chute}
    \PYG{n}{display}\PYG{o}{.}\PYG{n}{clear}\PYG{p}{(}\PYG{p}{)}
    \PYG{n}{y}\PYG{p}{,} \PYG{n}{x} \PYG{o}{=} \PYG{l+m+mi}{0}\PYG{p}{,} \PYG{l+m+mi}{0}
    \PYG{n}{display}\PYG{o}{.}\PYG{n}{set\PYGZus{}pixel}\PYG{p}{(}\PYG{n}{x}\PYG{p}{,} \PYG{n}{y}\PYG{p}{,} \PYG{l+m+mi}{9}\PYG{p}{)}
    \PYG{n}{sleep}\PYG{p}{(}\PYG{n}{t}\PYG{p}{)}
    \PYG{k}{while} \PYG{n}{y} \PYG{o}{\PYGZlt{}} \PYG{l+m+mi}{4}\PYG{p}{:}
        \PYG{n}{display}\PYG{o}{.}\PYG{n}{clear}\PYG{p}{(}\PYG{p}{)}
        \PYG{k}{if} \PYG{n+nb}{round}\PYG{p}{(}\PYG{n}{random}\PYG{p}{(}\PYG{p}{)}\PYG{p}{)}\PYG{p}{:}      \PYG{c+c1}{\PYGZsh{} si arrondi de alea est vrai (différent de 0)}
            \PYG{n}{y} \PYG{o}{=} \PYG{n}{y} \PYG{o}{+} \PYG{l+m+mi}{1}            \PYG{c+c1}{\PYGZsh{} on augmente y de 1}
        \PYG{k}{else}\PYG{p}{:}
            \PYG{n}{x} \PYG{o}{=} \PYG{n}{x} \PYG{o}{+} \PYG{l+m+mi}{1}
            \PYG{n}{y} \PYG{o}{=} \PYG{n}{y} \PYG{o}{+} \PYG{l+m+mi}{1}
        \PYG{n}{display}\PYG{o}{.}\PYG{n}{set\PYGZus{}pixel}\PYG{p}{(}\PYG{n}{x}\PYG{p}{,} \PYG{n}{y}\PYG{p}{,} \PYG{l+m+mi}{9}\PYG{p}{)}
        \PYG{n}{sleep}\PYG{p}{(}\PYG{n}{t}\PYG{p}{)}
    \PYG{n}{n}\PYG{p}{[}\PYG{n}{x}\PYG{p}{]} \PYG{o}{=} \PYG{n}{n}\PYG{p}{[}\PYG{n}{x}\PYG{p}{]}\PYG{o}{+}\PYG{l+m+mi}{1}        \PYG{c+c1}{\PYGZsh{} incrementation du compteur de la position x}
    \PYG{n}{display}\PYG{o}{.}\PYG{n}{set\PYGZus{}pixel}\PYG{p}{(}\PYG{n}{x}\PYG{p}{,} \PYG{n}{y}\PYG{p}{,} \PYG{l+m+mi}{1}\PYG{p}{)}


\PYG{k}{while} \PYG{k+kc}{True}\PYG{p}{:}
    \PYG{k}{if} \PYG{n}{button\PYGZus{}a}\PYG{o}{.}\PYG{n}{is\PYGZus{}pressed}\PYG{p}{(}\PYG{p}{)}\PYG{p}{:}
        \PYG{n}{chute}\PYG{p}{(}\PYG{l+m+mi}{500}\PYG{p}{)}

    \PYG{k}{elif} \PYG{n}{button\PYGZus{}b}\PYG{o}{.}\PYG{n}{get\PYGZus{}presses}\PYG{p}{(}\PYG{p}{)}\PYG{p}{:}
        \PYG{n}{n} \PYG{o}{=} \PYG{p}{[}\PYG{l+m+mi}{0}\PYG{p}{,} \PYG{l+m+mi}{0}\PYG{p}{,} \PYG{l+m+mi}{0}\PYG{p}{,} \PYG{l+m+mi}{0}\PYG{p}{,} \PYG{l+m+mi}{0}\PYG{p}{]}
        \PYG{k}{for} \PYG{n}{k} \PYG{o+ow}{in} \PYG{n+nb}{range}\PYG{p}{(}\PYG{l+m+mi}{80}\PYG{p}{)}\PYG{p}{:}
            \PYG{n}{chute}\PYG{p}{(}\PYG{n+nb}{round}\PYG{p}{(}\PYG{l+m+mi}{500} \PYG{o}{/} \PYG{p}{(}\PYG{l+m+mf}{1.05}\PYG{o}{*}\PYG{o}{*}\PYG{n}{k}\PYG{p}{)}\PYG{p}{)}\PYG{p}{)}
            \PYG{k}{for} \PYG{n}{j} \PYG{o+ow}{in} \PYG{n+nb}{range}\PYG{p}{(}\PYG{l+m+mi}{5}\PYG{p}{)}\PYG{p}{:}
                \PYG{n}{aff}\PYG{p}{(}\PYG{n}{n}\PYG{p}{[}\PYG{n}{j}\PYG{p}{]}\PYG{p}{,} \PYG{n}{j}\PYG{p}{)}
            \PYG{n}{sleep}\PYG{p}{(}\PYG{l+m+mi}{200}\PYG{p}{)}
        \PYG{n+nb}{print}\PYG{p}{(}\PYG{n}{n}\PYG{p}{)}
\end{sphinxVerbatim}


\section{Pierrot et Simon}
\label{\detokenize{projets/pierrot:projetpierrot}}\label{\detokenize{projets/pierrot:pierrot-et-simon}}\label{\detokenize{projets/pierrot::doc}}

\subsection{Description}
\label{\detokenize{projets/pierrot:description}}
\begin{sphinxadmonition}{note}{\label{projets/pierrot:index-0}À faire:}
capture d’écran / gif animée
\end{sphinxadmonition}


\subsubsection{Exemple(s) d’utilisation}
\label{\detokenize{projets/pierrot:exemple-s-d-utilisation}}

\paragraph{Escape game}
\label{\detokenize{projets/pierrot-exemple-escape:escape-game}}\label{\detokenize{projets/pierrot-exemple-escape::doc}}
Nous avons utilisé le projet {\hyperref[\detokenize{projets/pierrot:projetpierrot}]{\sphinxcrossref{\DUrole{std,std-ref}{Pierrot et Simon}}}} pour un escape
game proposé en stage.
\begin{itemize}
\item {} 
diaporama d’acceuil : \sphinxurl{http://url.univ-irem.fr/pierrot}

\item {} 
page de formation : \sphinxurl{http://url.univ-irem.fr/algo1718-pierrot}

\end{itemize}


\subsection{Réalisation}
\label{\detokenize{projets/pierrot:realisation}}

\subsubsection{Fabriquer}
\label{\detokenize{projets/pierrot-fabriquer::doc}}\label{\detokenize{projets/pierrot-fabriquer:fabriquer}}
Nous détaillons ici comment fabriquer et assembler
le matériel nécessaire à la réalisation du projet
{\hyperref[\detokenize{projets/pierrot:projetpierrot}]{\sphinxcrossref{\DUrole{std,std-ref}{Pierrot et Simon}}}}.

\begin{sphinxadmonition}{note}{\label{projets/pierrot-fabriquer:index-0}À faire:}
tout faire.
\end{sphinxadmonition}


\subsubsection{Coder}
\label{\detokenize{projets/pierrot-coder:coder}}\label{\detokenize{projets/pierrot-coder::doc}}
Nous détaillons ici le code nécessaire à la réalisation
du projet {\hyperref[\detokenize{projets/pierrot:projetpierrot}]{\sphinxcrossref{\DUrole{std,std-ref}{Pierrot et Simon}}}}.

\begin{sphinxadmonition}{note}{\label{projets/pierrot-coder:index-0}À faire:}
tout à faire!
\end{sphinxadmonition}


\chapter{Index et page de recherche}
\label{\detokenize{index:index-et-page-de-recherche}}\begin{itemize}
\item {} 
\DUrole{xref,std,std-ref}{genindex}

\item {} 
\DUrole{xref,std,std-ref}{search}

\end{itemize}


\bigskip\hrule\bigskip

\sphinxhref{http://microbit.readthedocs.io/fr/latest/?badge=latest}{}


\renewcommand{\indexname}{Index}
\printindex
\end{document}